\documentclass[12pt]{article}
\usepackage{array}
\usepackage{multicol}
\usepackage{amssymb} %math symbols
\usepackage{amsmath} %math stuff
\usepackage{mathrsfs}
\usepackage{fullpage}
\usepackage{epstopdf}
\usepackage{setspace} %Spacing
\usepackage{graphicx}
\usepackage{enumerate}
\usepackage{gensymb}
\usepackage{beamerarticle} %uncomment when making beamernotes
\usepackage{textcomp}
\usepackage{amsfonts}

\usepackage{tikz}
\usetikzlibrary{shapes,snakes,positioning}

\begin{document}
\title{\textmd{Sasybel}: Translations}
\author{
        Marie-Andree B.Langlois \\
        \texttt{marie-andree.blanglois@mail.mcgill.ca}
}
\date{\today}
\maketitle

\begin{abstract}
This document monitors the development of Sasybel and it's theorem translation achievements.
\end{abstract}

\section{September 29 2011}
The following \textmd{Sasybel} code:
\begin{verbatim}

theorem zless: exists z < (s z) ;

dsolve : z < (s z) by rule less-one

end theorem

\end{verbatim}
was translated and produced the following output was produced after the reconstruction:
\begin{verbatim}

## SASyLF parsing: nat.slf ##

rec zless : (less z (s z))[] = 
[] l_one z

## Type Reconstruction done: nat.slf  ##

\end{verbatim}
This is the same theorem written directly in \textmd{Beluga}.
\begin{verbatim}

## Type Reconstruction: sum1.bel ##

rec zless : (less z (s z))[] = 
[] l_one z

## Type Reconstruction done: sum1.bel  ##
\end{verbatim}

\section{November 15 2011}
The following \textmd{Sasybel} code:
\begin{verbatim}
theorem plus1 : forall N exists N + (s z) = (s N);

x : N + (s z) = (s N) by induction on N : 

case z is
	   d2: (z) + (s z) = (s z) 	by rule sum-z
end case

end induction
end theorem 

\end{verbatim}
was translated and produced the following output was produced after the reconstruction:
\begin{verbatim}

## SASyLF parsing: sum2.slf ##

rec plus1 : {N :: n[]} (sum N (s z) (s N))[] = 
mlam N =>  case [] N : n[] of 
           | ([] z) :  . z = N ;   => 
              [] sum-z (s z)
           

## Type Reconstruction done: sum2.slf  ##

\end{verbatim}
This is the same theorem written directly in \textmd{Beluga}.
\begin{verbatim}

## Type Reconstruction: sum.bel ##

rec plus1 : {N :: nat[]} (add N (s z) (s N))[] = 
mlam N =>  case [] N : nat[] of 
           | ([] z) :  . z = N ;   => 
              [] a_z (s z)
           

## Type Reconstruction done: sum.bel  ##

\end{verbatim}

\section{November 24 2011}
The following \textmd{Sasybel} code:
\begin{verbatim}

theorem plus1 : forall N exists N + (s z) = (s N);

x : N + (s z) = (s N) by induction on N : 

case z is
	   d2: (z) + (s z) = (s z) 	by rule sum-z
end case

case (s N) is
	               F : N + (s z) = (s N) by induction hypothesis on N
                       d4 : (s N) + (s N) = (s (s N))  by rule sum-s on F
end case

end induction
end theorem 

\end{verbatim}
was translated and produced the following output was produced after the reconstruction:
\begin{verbatim}

## SASyLF parsing: sum2.slf ##

rec plus1 : {N :: n[]} (sum N (s z) (s N))[] = 
mlam N => 
  case [] N : n[] of 
  | ([] z) :  . z = N ;   => 
     [] sum-z (s z)
  
  | {Y :: n[]}
    ([] s Y) :  . s Y = N ;   => 
      (case plus1 <. Y> of 
       | {Z1 :: n[]}  {X1 :: (sum Z1 (s z) (s Z1))[]}
         ([] X1) :  . Z1 = Y ;   => 
          [] sum-s Z1 (s z) (s Z1) X1
       )
  

## Type Reconstruction done: sum2.slf  ##

\end{verbatim}
This is the same theorem written directly in \textmd{Beluga}.
\begin{verbatim}

## Type Reconstruction: sum.bel ##

rec plus1 : {N :: nat[]} (add N (s z) (s N))[] = 
mlam N => 
  case [] N : nat[] of 
  | ([] z) :  . z = N ;   => 
     [] a_z (s z)
  
  | {Y :: nat[]}
    ([] s Y) :  . s Y = N ;   => 
      (case plus1 <. Y> of 
       | {Z1 :: nat[]}  {X1 :: (add Z1 (s z) (s Z1))[]}
         ([] X1) :  . Z1 = Y ;   => 
          [] a_s Z1 (s z) (s Z1) X1
       )
  

## Type Reconstruction done: sum.bel  ##
\end{verbatim}
\section{January 09 2012}
The following \textmd{Sasybel} code:
\begin{verbatim}

theorem sum-s-rh : forall D1 : N1 + N2 = N3 exists N1 + (s N2) = (s N3);

d2 : N1 + (s N2) = (s N3) by induction on D1 :

case rule

--------------------- sum-z
dzc : (z) + N = N ;

is

dz1 : (z) + (s N) = (s N) by rule sum-z

end case
end induction
end theorem 

\end{verbatim}
was translated and produced the following output was produced after the reconstruction:
\begin{verbatim}

## SASyLF parsing: sum2.slf ##
rec sum-s-rh : {N1 :: n[]} {N2 :: n[]} {N3 :: n[]} {D1 :: (sum N1 N2 N3)[]}
                 (sum N1 (s N2) (s N3))[] = 
mlam N1 => mlam N2 => mlam N3 => mlam D1 => 
  case [] D1 : (sum N1 N2 N3)[] of 0
  | {X :: n[]}
    ([] sum-z X) :  . sum-z X = D1 ,  . X = N3 ,  . X = N2 ,  . z = N1 ;   => 
     [] sum-z (s X)
  

## Type Reconstruction done: sum2.slf  ##


\end{verbatim}
This is the same theorem written directly in \textmd{Beluga}.
\begin{verbatim}

## Type Reconstruction: sum.bel ##
z : nat .
s : nat -> nat .
a_z : {R : nat}  add z R R .
a_s : {N1 : nat}  {N2 : nat}  {N3 : nat}  add N1 N2 N3 -> add (s N1) N2 (s N3) .

rec sumsr : {Nat1 :: nat[]} {Nat2 :: nat[]} {Nat3 :: nat[]}
              {D1 :: (add Nat1 Nat2 Nat3)[]} (add Nat1 (s Nat2) (s Nat3))[] = 
mlam Nat1 => mlam Nat2 => mlam Nat3 => mlam D1 => 
  case [] D1 : (add Nat1 Nat2 Nat3)[] of 
  | {X :: nat[]}
    ([] a_z X) :  . a_z X = D1 ,  . X = Nat3 ,  . X = Nat2 ,  . z = Nat1 ;   => 
     [] a_z (s X)
  

## Type Reconstruction done: sum.bel  ##

\end{verbatim}

\section{January 20 2012}
The following \textmd{Sasybel} code:
\footnotesize\begin{verbatim}

theorem sum-s-rh : forall D1 : N1 + N2 = N3 exists N1 + (s N2) = (s N3);

d2 : N1 + (s N2) = (s N3) by induction on D1 :

case rule

--------------------- sum-z
z + N = N;

is

dz1 : z + (s N) = (s N) by rule sum-z

end case


case rule

I : N1' + N2 = N3';
---------------------------- sum-s
(s N1') + N2 = (s N3');

is

H : N1' + (s N2) = (s N3')  by induction hypothesis on I
(s N1') + (s N2) = (s s N3') by rule sum-s on H

end case 
end induction
end theorem 

\end{verbatim}
was translated and produced the following output was produced after the reconstruction:
\footnotesize\begin{verbatim}

## SASyLF parsing: sum2.slf ##
rec sum-s-rh : {N1 :: n[]} {N2 :: n[]} {N3 :: n[]} {D1 :: (sum N1 N2 N3)[]}
                 (sum N1 (s N2) (s N3))[] = 
mlam N1 => mlam N2 => mlam N3 => mlam D1 => 
  case [] D1 : (sum N1 N2 N3)[] of 
  | {X :: n[]}
    ([] sum-z X) :  . sum-z X = D1 ,  . X = N3 ,  . X = N2 ,  . z = N1 ;   => 
     [] sum-z (s X)
  
  | {Y2 :: n[]}  {Z1 :: n[]}  {X1 :: n[]}  {Y1 :: (sum Y2 Z1 X1)[]}
    ([] sum-s Y2 Z1 X1 Y1) :  . sum-s Y2 Z1 X1 Y1 = D1 ,  . s X1 = N3 ,  . Z1 = N2 ,
     . s Y2 = N1 ;   => 
      (case sum-s-rh <. Y2> <. Z1> <. X1> <. Y1> of 
       | {Z4 :: n[]}  {X4 :: n[]}  {Y4 :: n[]}  {Z3 :: (sum Z4 X4 Y4)[]} 
         {X3 :: (sum Z4 (s X4) (s Y4))[]}
         ([] X3) :  . Z3 = Y1 ,  . Y4 = X1 ,  . X4 = Z1 ,  . Z4 = Y2 ;   => 
          [] sum-s Z4 (s X4) (s Y4) X3
       )
  

## Type Reconstruction done: sum2.slf  ##



\end{verbatim}
This is the same theorem written directly in \textmd{Beluga}.
\footnotesize\begin{verbatim}

## Type Reconstruction: sum.bel ##
rec sumsr : {Nat1 :: nat[]} {Nat2 :: nat[]} {Nat3 :: nat[]}
              {D1 :: (add Nat1 Nat2 Nat3)[]} (add Nat1 (s Nat2) (s Nat3))[] = 
mlam Nat1 => mlam Nat2 => mlam Nat3 => mlam D1 => 
  case [] D1 : (add Nat1 Nat2 Nat3)[] of 
  | {X :: nat[]}
    ([] a_z X) :  . a_z X = D1 ,  . X = Nat3 ,  . X = Nat2 ,  . z = Nat1 ;   => 
     [] a_z (s X)
  
  | {Y2 :: nat[]}  {Z1 :: nat[]}  {X1 :: nat[]}  {Y1 :: (add Y2 Z1 X1)[]}
    ([] a_s Y2 Z1 X1 Y1) :  . a_s Y2 Z1 X1 Y1 = D1 ,  . s X1 = Nat3 ,  . Z1 = Nat2 ,
     . s Y2 = Nat1 ;   => 
      (case sumsr <. Y2> <. Z1> <. X1> <. Y1> of 
       | {Z4 :: nat[]}  {X4 :: nat[]}  {Y4 :: nat[]}  {Z3 :: (add Z4 X4 Y4)[]} 
         {X3 :: (add Z4 (s X4) (s Y4))[]}
         ([] X3) :  . Z3 = Y1 ,  . Y4 = X1 ,  . X4 = Z1 ,  . Z4 = Y2 ;   => 
          [] a_s Z4 (s X4) (s Y4) X3
       )
  

## Type Reconstruction done: sum.bel  ##


\end{verbatim}

\section{February 1 2012}
The following \textmd{Sasybel} code:
\footnotesize\begin{verbatim}

Context gamma ::= () | {y:exp, t:tp} y oft t;

\end{verbatim}
was translated and produced the following output was produced after the reconstruction:
\footnotesize\begin{verbatim}

## Sasybel Type Reconstruction: typuni.slf ##

schema gamma = some [y : exp, t : tp] of_typ y t;

## Type Reconstruction done: typuni.slf  ##

\end{verbatim}
This is the same theorem written directly in \textmd{Beluga}.
\footnotesize\begin{verbatim}
## Type Reconstruction: typuni.bel ##

schema tctx = some [t : tp, x : exp] type_of x t;

## Type Reconstruction done: typuni.bel  ##

\end{verbatim}

\section{March 29 2012}
The following \textmd{Sasybel} code:
\footnotesize\begin{verbatim}

context gamma ::= {t:tp} (x:exp) x oft t;

judgment equal : tp = tp;

------------------------- eq_ref
T = T;

theorem unique : forall d: [g: gamma] |- E .. oft T ; f: [g] |- E .. oft T' exists T = T';

T = T' by induction on d :

case [g: (x:exp) x oft T] |- x oft T

is

T = T' by case analysis on f:

                 case [g: (y:exp) y oft T'] |- y oft T'

                 is 

                 eq_ref  by uniqueness of variables

                 end case
                 end case analysis

end case

end induction
end theorem

\end{verbatim}
was translated and produced the following output was produced after the reconstruction:
\footnotesize\begin{verbatim}

## Sasybel Type Reconstruction: typuni.sbel ##

schema gamma = some [t : tp] block (x:exp. of_typ x t);
equal : tp -> tp -> type .
eq_ref : {T : tp}  equal T T .

rec unique : {g:(gamma)*} {E :: exp[g]}^i {T :: tp[]}^i {T' :: tp[]}^i
               (of_typ (E ..) (T ))[g] -> (of_typ (E ..) (T' ))[g] -> (equal T T')[] = 
FN g => mlam E => mlam T => mlam T' => fn f => fn d => 
  case d of 
  | {y6 :: (some [t : tp] block (x:exp. of_typ x t))*}  {Z :: tp[]} 
    {#z5 :: block (x:exp. of_typ x (Z ))[y6]}  {X :: tp[]}
    [[y6. #z5.2 ..]  :  . Z = T' ,  . X = T , y6 . (#z5.1 ..) = E , y6 = g ]   => 
      (case f of 
       | {z6 :: (some [t : tp] block (x:exp. of_typ x t))*}  {X1 :: tp[]} 
         {#x6 :: block (x:exp. of_typ x (X1 ))[z6]}
         [[z6. #x6.2 ..]  :  . X1 = X , z6 . (#x6 ..) = #z5 ,  . X1 = Z ,
         z6 = y6 ]   => 
          [] eq_ref X1 : (equal X1 X1)[]
       )
  

## Type Reconstruction done: typuni.sbel  ##

\end{verbatim}
This is the same theorem written directly in \textmd{Beluga}.
\footnotesize\begin{verbatim}
## Type Reconstruction: unique.bel ##

rec unique : {g:(tctx)*} {E :: exp[g]}^i {T :: tp[]}^i {T' :: tp[]}^i
               (type_of (E ..) (T ))[g] -> (type_of (E ..) (T' ))[g] -> (equal T T')[] = 
FN g => mlam E => mlam T => mlam T' => fn d => fn f => 
  case d of 
  | {x5 :: (some [t : tp] block (x:exp. type_of x t))*}  {Z :: tp[]} 
    {#H :: block (x:exp. type_of x (Z ))[x5]}  {X :: tp[]}
    [[x5. #H.2 ..]  :  . X = T' ,  . Z = T , x5 . (#H.1 ..) = E , x5 = g ]   => 
      (case f of 
       | {z5 :: (some [t : tp] block (x:exp. type_of x t))*}  {X1 :: tp[]} 
         {#H1 :: block (x:exp. type_of x (X1 ))[z5]}
         [[z5. #H1.2 ..]  :  . X1 = X , z5 . (#H1 ..) = #H ,  . X1 = Z ,
         z5 = x5 ]   => 
          [] e_ref X1 : (equal X1 X1)[]
       )
  

## Type Reconstruction done: unique.bel  ##

\end{verbatim}

\section{April 10 2012}
The following \textmd{Sasybel} code:
\footnotesize\begin{verbatim}

terminals z suc lambda app
syntax 

exp ::= z
| suc exp
| x
| lambda \x .exp[x] 
| app exp exp 
| let exp = x in exp[x] ;

notation value : exp value ;

--------------- val-z
z value ;

e1 value ;
--------------- val-s
(suc e1) value ;

------------------------------ val-lam
(lambda \x . e1[x]) value ;

notation eval : exp eval exp;

--------------- ev-z
z eval z;

e1 eval e2;
------------------------- ev-s
(suc e1) eval (suc e2);

--------------------------------------------- ev-lam
(lambda \x .e1[x]) eval (lambda \x .e1[x]);

e1 eval (lambda \x .e[x]);
e2 eval e2';
e[e2'] eval e1';
----------------------------------------- ev-app
(app e1 e2) eval e1';

(e2[e1]) eval v;
e1 eval v1;
------------------------------------------- ev-let
(let e1 = x in e2[x]) eval v;



theorem val-sound : forall D: E eval E' exists E' value ;

E' value by induction on D:

case rule

	--------------- ev-z
	z eval z;

	is

	E' value by rule val-z

end case

case rule 

	I: e1 eval e2;
	---------------------------- ev-s
	(suc e1) eval (suc e2);

	is

	H: e2 value by induction hypothesis on I;
	e' value by rule val-s on H;

end case

case rule 

	-------------------------------------------------------- ev-lam
	(lambda \x . e1[x]) eval (lambda \x . e1[x]);

	is

	e' value by rule val-lam

end case

case rule 

	D1: e1 eval lambda \x . e3[x];
	D2: e2 eval e2';
	D3: e3[e2'] eval e1';
	------------------------------------- ev-app
	(e1 e2) eval e1';

	is

	e' value by induction hypothesis on D3;

end case

case rule 

	I: e2[e1] eval e1';
	H: e1 eval e1';
	----------------------------------------------- ev-let
	(let x=e1 in e2[x]) eval e1';

	is

	e' value by induction hypothesis on I;

end case

end induction 
end theorem

\end{verbatim}
was translated and produced the following output was produced after the reconstruction:
\footnotesize\begin{verbatim}

## Sasybel Type Reconstruction: vs.sbel ##

z : exp .
suc : exp -> exp .
lambda : (exp -> exp) -> exp .
app : exp -> exp -> exp .
letv : exp -> (exp -> exp) -> exp .
value : exp -> type .
val-z : value z .
val-s : {e1 : exp}  value e1 -> value (suc e1) .
val-lam : {e1 : {x3 : exp}  exp}  value (lambda (\x. e1 x)) .
eval : exp -> exp -> type .
ev-z : eval z z .
ev-s : {e1 : exp}  {e2 : exp}  eval e1 e2 -> eval (suc e1) (suc e2) .
ev-lam : {e1 : {z4 : exp}  exp}  eval (lambda (\x. e1 x)) (lambda (\x. e1 x)) .
ev-app : {e1 : exp} 
 {e : {y6 : exp}  exp} 
  {e2 : exp} 
   {e2' : exp} 
    {e1' : exp} 
     eval e1 (lambda (\x. e x)) -> eval e2 e2' -> eval (e e2') e1' -> eval (app e1 e2) e1' .
ev-let : {e2 : {y7 : exp}  exp} 
 {e1 : exp} 
  {v : exp} 
   {v1 : exp}  eval (e2 e1) v -> eval e1 v1 -> eval (letv e1 (\x. e2 x)) v .

              

rec val-sound : {E :: exp[]}^i {E' :: exp[]}^i
                                 {D :: (eval E E')[]}^e (value E')[] = 
mlam E => mlam E' => mlam D => 
  case [] D : (eval E E')[] of 
  | [[. ev-z]  :  . ev-z = D ,  . z = E' ,  . z = E ]   => 
     [] val-z : (value z)[]
  
  | {X1 :: exp[]}  {Y1 :: exp[]}  {Z :: (eval X1 Y1)[]}
    [[. ev-s X1 Y1 Z]  :  . ev-s X1 Y1 Z = D ,  . suc Y1 = E' ,
     . suc X1 = E ]   => 
      (case val-sound <. X1> <. Y1> <. Z> of 
       | {Y3 :: exp[]}  {Z2 :: exp[]}  {X2 :: (eval Y3 Z2)[]} 
         {Y2 :: (value Z2)[]}
         [[. Y2]  :  . X2 = Z ,  . Z2 = Y1 ,  . Y3 = X1 ]   => 
          [] val-s Z2 Y2
       )
  
  | {X :: exp[z4 : exp]}
    [[. ev-lam (\z4. X)]  :  . ev-lam (\z4. X) = D ,  . lambda (\x. X x) = E' ,
     . lambda (\x. X x) = E ]   => 
     [] val-lam (\x3. X x3) : (value (lambda (\x. X x)))[]
  
  | {Y4 :: exp[]}  {Z3 :: exp[]}  {X3 :: exp[]}  {Y3 :: exp[y6 : exp]} 
    {Z2 :: exp[]}  {X2 :: (eval Y4 (lambda (\x. Y3 x)))[]} 
    {Y2 :: (eval Z3 Z2)[]}  {Z1 :: (eval (Y3 Z2) X3)[]}
    [[. ev-app Y4 (\y6. Y3) Z3 Z2 X3 X2 Y2 Z1]  :  . ev-app Y4 (\y6. Y3) Z3 Z2 X3 X2 Y2 Z1 = D
    ,  . X3 = E' ,  . app Y4 Z3 = E ]   => 
     val-sound <. Y3 Z2> <. X3> <. Z1>
  
  | {Y3 :: exp[]}  {Z2 :: exp[y7 : exp]}  {X2 :: exp[]}  {Y2 :: exp[]} 
    {Z1 :: (eval (Z2 Y3) X2)[]}  {X1 :: (eval Y3 Y2)[]}
    [[. ev-let (\y7. Z2) Y3 X2 Y2 Z1 X1]  :  . ev-let (\y7. Z2) Y3 X2 Y2 Z1 X1 = D
    ,  . X2 = E' ,  . letv Y3 (\x. Z2 x) = E ]   => 
     val-sound <. Z2 Y3> <. X2> <. Z1>
  

## Type Reconstruction done: vs.sbel  ##

\end{verbatim}
This is the same theorem written directly in \textmd{Beluga}.
\footnotesize\begin{verbatim}
## Type Reconstruction: examples/mini-ml/vsound.bel ##
exp : type .
z : exp .
suc : exp -> exp .
letv : exp -> (exp -> exp) -> exp .
lam : (exp -> exp) -> exp .
app : exp -> exp -> exp .
value : exp -> type .
v_z : value z .
v_s : {E : exp}  value E -> value (suc E) .
v_lam : {E : {x3 : exp}  exp}  value (lam (\x. E x)) .
eval : exp -> exp -> type .
ev_z : eval z z .
ev_s : {E : exp}  {V : exp}  eval E V -> eval (suc E) (suc V) .
ev_l : {E2 : {x5 : exp}  exp} 
 {V1 : exp} 
  {V : exp} 
   {E1 : exp}  eval (E2 V1) V -> eval E1 V1 -> eval (letv E1 (\x. E2 x)) V .
ev_lam : {E : {z5 : exp}  exp}  eval (lam (\x. E x)) (lam (\x. E x)) .
ev_app : {E1 : exp} 
 {E : {y7 : exp}  exp} 
  {E2 : exp} 
   {V2 : exp} 
    {V : exp} 
     eval E1 (lam (\x. E x)) -> eval E2 V2 -> eval (E V2) V -> eval (app E1 E2) V .

rec vs : {E :: exp[]}^i {V :: exp[]}^i (eval E V)[] -> (value V)[] = 
mlam E => mlam V => fn e => 
  case e of 
  | [[. ev_z]  :  . z = V ,  . z = E ]   => 
     [] v_z : (value z)[]
  
  | {X1 :: exp[]}  {Y1 :: exp[]}  {Z :: (eval X1 Y1)[]}
    [[. ev_s X1 Y1 Z]  :  . suc Y1 = V ,  . suc X1 = E ]   => 
      (case vs <. X1> <. Y1> ([] Z) of 
       | {Y3 :: exp[]}  {Z2 :: exp[]}  {X2 :: (eval Y3 Z2)[]} 
         {Y2 :: (value Z2)[]}
         [[. Y2]  :  . X2 = Z ,  . Z2 = Y1 ,  . Y3 = X1 ]   => 
          [] v_s Z2 Y2 : (value (suc Z2))[]
       )
  
  | {Y3 :: exp[]}  {Z2 :: exp[x5 : exp]}  {X2 :: exp[]}  {Y2 :: exp[]} 
    {Z1 :: (eval (Z2 Y2) X2)[]}  {X1 :: (eval Y3 Y2)[]}
    [[. ev_l (\x5. Z2) Y2 X2 Y3 Z1 X1]  :  . X2 = V ,
     . letv Y3 (\x. Z2 x) = E ]   => 
     vs <. Z2 Y2> <. X2> ([] Z1)
  
  | {X :: exp[z5 : exp]}
    [[. ev_lam (\z5. X)]  :  . lam (\x. X x) = V ,  . lam (\x. X x) = E ]   => 
     [] v_lam (\x3. X x3) : (value (lam (\x. X x)))[]
  
  | {Y4 :: exp[]}  {Z3 :: exp[]}  {X3 :: exp[]}  {Y3 :: exp[y7 : exp]} 
    {Z2 :: exp[]}  {X2 :: (eval Y4 (lam (\x. Y3 x)))[]} 
    {Y2 :: (eval Z3 Z2)[]}  {Z1 :: (eval (Y3 Z2) X3)[]}
    [[. ev_app Y4 (\y7. Y3) Z3 Z2 X3 X2 Y2 Z1]  :  . X3 = V ,
     . app Y4 Z3 = E ]   => 
     vs <. Y3 Z2> <. X3> ([] Z1)
  

## Type Reconstruction done: examples/mini-ml/vsound.bel  ##

\end{verbatim}

\section{April 13 2012}
The two following cases of the type uniqueness proof, from \textmd{Sasybel} code:
\footnotesize\begin{verbatim}

case rule

D1: [g] |- E1 oft (T2 -> T);  
D2: [g] |- E2 oft T2;
------------------------------ t_app
[g] |- (app E1 E2) oft T;

is

eq_ref by case analysis on f:

                 case rule

                 F1: [g] |- E1 oft (T2 -> T');  
                 F2: [g] |- E2 oft T2;
                 ------------------------------ t_app
                 [g] |- (app E1 E2) oft T';

                 is 

                 eq_ref by induction hypothesis on [g] |- D1 .. , [g] |- F1 .. ;

                 end case
                 end case analysis

T = T' by rule eq_ref

end case

case rule

D: [g] |- {x:exp} (x oft T1) (E[x] oft T2);
-------------------------------------------- t_lam
[g] |- (lam T1 E) oft (T1 -> T2);

is

eq_ref by case analysis on f:

                 case rule


                 F: [g] |- {x:exp} (x oft T1) (E[x] oft T2);
                 -------------------------------------------- t_lam
                 [g] |- (lam T1 E) oft (T1 -> T2);

                 is 

                 eq_ref  by induction hypothesis on [g, b:(x:exp) x oft _] |- D .. b.1 b.2 , 
                                                    [g,b] |- F .. b.1 b.2 ;

                 end case
                 end case analysis

T = T' by rule eq_ref

\end{verbatim}
was translated and produced the following output was produced after the reconstruction:
\footnotesize\begin{verbatim}

## Sasybel Type Reconstruction: examples/sasybel/unique.sbel ##

  case d of 
  | {y6 :: (some [t : tp] block (x:exp. of_typ x t))*}  {X3 :: exp[y6]} 
    {Y3 :: exp[y6]}  {Z2 :: tp[]}  {X2 :: tp[]}  {Y2 :: tp[y6]} 
    {Z1 :: (of_typ (X3 ..) (arr (Y2 ..) (X2 )))[y6]} 
    {X1 :: (of_typ (Y3 ..) (Y2 ..))[y6]}
    [[y6. t_app (X3 ..) (Y2 ..) (X2 ) (Y3 ..) (Z1 ..) (X1 ..)]  :  . X2 = T' ,
     . Z2 = T , y6 . app (X3 ..) (Y3 ..) = E , y6 = g ]   => 
      (case f of 
       | {x6 :: (some [t : tp] block (x:exp. of_typ x t))*}  {Y8 :: exp[x6]} 
         {Z7 :: exp[x6]}  {X7 :: tp[]}  {Y7 :: tp[]}  {Z6 :: tp[x6]} 
         {X6 :: (of_typ (Y8 ..) (arr (Z6 ..) (Y7 )))[x6]} 
         {Y6 :: (of_typ (Z7 ..) (Z6 ..))[x6]}  {Z5 :: tp[x6]} 
         {X5 :: (of_typ (Y8 ..) (arr (Z5 ..) (X7 )))[x6]} 
         {Y5 :: (of_typ (Z7 ..) (Z5 ..))[x6]}
         [[x6. t_app (Y8 ..) (Z5 ..) (X7 ) (Z7 ..) (X5 ..) (Y5 ..)]  : x6 . (Y6 ..) = X1
         , x6 . (X6 ..) = Z1 , x6 . (Z6 ..) = Y2 ,  . Y7 = X2 ,  . X7 = Z2 ,
         x6 . (Z7 ..) = Y3 , x6 . (Y8 ..) = X3 , x6 = y6 ]   => 
           (case unique [x6] <x6. Y8 ..> <. arr Z6 Y7> <. arr Z5 X7>
                   ([x6] X6 ..) ([x6] X5 ..) of 
            | {z6 :: (some [t : tp] block (x:exp. of_typ x t))*} 
              {Y11 :: exp[z6]}  {Z10 :: exp[z6]}  {X10 :: tp[]} 
              {Y10 :: tp[z6]} 
              {Z9 :: (of_typ (Y11 ..) (arr (Y10 ..) (X10 )))[z6]} 
              {X9 :: (of_typ (Z10 ..) (Y10 ..))[z6]} 
              {Y9 :: (of_typ (Y11 ..) (arr (Y10 ..) (X10 )))[z6]} 
              {Z8 :: (of_typ (Z10 ..) (Y10 ..))[z6]}
              [[. eq_ref (arr Y10 X10)]  : z6 . (Z8 ..) = Y5 ,
              z6 . (Y9 ..) = X5 , z6 . (Y10 ..) = Z5 , z6 . (X9 ..) = Y6 ,
              z6 . (Z9 ..) = X6 , z6 . (Y10 ..) = Z6 ,  . X10 = Y7 ,
               . X10 = X7 , z6 . (Z10 ..) = Z7 , z6 . (Y11 ..) = Y8 ,
              z6 = x6 ]   => 
               [] eq_ref X10 : (equal X10 X10)[]
            )
       )

  | {y9 :: (some [t : tp] block (x:exp. of_typ x t))*}  {X2 :: tp[]} 
    {Y2 :: exp[y9, z4 : exp]}  {Z1 :: tp[]}  {X1 :: tp[]} 
    {Y1 :: (of_typ (Y2 .. x) (X1 ))[y9, x : exp, x4 : of_typ x (X2 )]}
    [[y9. t_lam (X2 ) (\z4. Y2 ..) (X1 ) (\x. \u. Y1 .. x u)]  :  . arr X2 X1 = T'
    ,  . Z1 = T , y9 . lam (X2 ) (\z1. Y2 .. z1) = E , y9 = g ]   => 
      (case f of 
       | {z9 :: (some [t : tp] block (x:exp. of_typ x t))*}  {X5 :: tp[]} 
         {Y5 :: exp[z9, z4 : exp]}  {Z4 :: tp[]}  {X4 :: tp[]} 
         {Y4 :: (of_typ (Y5 .. x) (X4 ))[z9, x : exp, x4 : of_typ x (X5 )]} 
         {Z3 :: (of_typ (Y5 .. x) (Z4 ))[z9, x : exp, x4 : of_typ x (X5 )]}
         [[z9. t_lam (X5 ) (\z4. Y5 ..) (Z4 ) (\x. \u. Z3 .. x u)]  : z9, y11, z10 . (Y4 ..) = Y1
         ,  . X4 = X1 ,  . arr X5 Z4 = Z1 , z9, x11 . (Y5 .. x11) = Y2 ,
          . X5 = X2 , z9 = y9 ]   => 
           (case unique [z9, b : block (x:exp. of_typ x (X5 ))]
                   <z9, z11. Y5 .. z11.1> <. X4> <. Z4>
                   ([z9, y12] Y4 .. y12.1 y12.2) ([z9, x12] Z3 .. x12.1 x12.2) of 
            | {x10 :: (some [t : tp] block (x:exp. of_typ x t))*} 
              {X7 :: tp[]}  {Y7 :: exp[x10, z4 : exp]}  {Z6 :: tp[]} 
              {X6 :: (of_typ (Y7 .. x) (Z6 ))[x10, x : exp, x4 : of_typ x (X7 )]} 
              {Y6 :: (of_typ (Y7 .. x) (Z6 ))[x10, x : exp, x4 : of_typ x (X7 )]}
              [[. eq_ref Z6]  : x10, y13, z12 . (Y6 ..) = Z3 ,
              x10, z13, x13 . (X6 ..) = Y4 ,  . Z6 = X4 ,  . Z6 = Z4 ,
              x10, y14 . (Y7 ..) = Y5 ,  . X7 = X5 , x10 = z9 ]   => 
               [] eq_ref (arr X7 Z6) : (equal (arr X7 Z6) (arr X7 Z6))[]
            )
       )
  

## Type Reconstruction done: examples/sasybel/unique.sbel  ##


\end{verbatim}
This is the same theorem written directly in \textmd{Beluga}.
\footnotesize\begin{verbatim}
## Type Reconstruction: examples/unique/unique.bel ##

  case d of 
  | {x5 :: (some [t : tp] block (x:exp. type_of x t))*}  {E4 :: exp[x5]} 
    {E3 :: exp[x5]}  {T4 :: tp[]}  {T3 :: tp[]}  {T2 :: tp[x5]} 
    {H1 :: (type_of (E4 ..) (arr (T2 ..) (T4 )))[x5]} 
    {H :: (type_of (E3 ..) (T2 ..))[x5]}
    [[x5. t_app (E4 ..) (T2 ..) (T4 ) (E3 ..) (H1 ..) (H ..)]  :  . T3 = T' ,
     . T4 = T , x5 . app (E4 ..) (E3 ..) = E , x5 = g ]   => 
      (case f of 
       | {z5 :: (some [t : tp] block (x:exp. type_of x t))*}  {E8 :: exp[z5]} 
         {E7 :: exp[z5]}  {T10 :: tp[]}  {T9 :: tp[]}  {T8 :: tp[z5]} 
         {H5 :: (type_of (E8 ..) (arr (T8 ..) (T10 )))[z5]} 
         {H4 :: (type_of (E7 ..) (T8 ..))[z5]}  {T7 :: tp[z5]} 
         {H3 :: (type_of (E8 ..) (arr (T7 ..) (T9 )))[z5]} 
         {H2 :: (type_of (E7 ..) (T7 ..))[z5]}
         [[z5. t_app (E8 ..) (T7 ..) (T9 ) (E7 ..) (H3 ..) (H2 ..)]  : z5 . (H4 ..) = H
         , z5 . (H5 ..) = H1 , z5 . (T8 ..) = T2 ,  . T9 = T3 ,  . T10 = T4 ,
         z5 . (E7 ..) = E3 , z5 . (E8 ..) = E4 , z5 = x5 ]   => 
           (case unique3 [z5] <z5. E8 ..> <. arr T8 T10> <. arr T7 T9>
                   ([z5] H5 ..) ([z5] H3 ..) of 
            | {y6 :: (some [t : tp] block (x:exp. type_of x t))*} 
              {E10 :: exp[y6]}  {E9 :: exp[y6]}  {T13 :: tp[]} 
              {T12 :: tp[y6]} 
              {H9 :: (type_of (E10 ..) (arr (T12 ..) (T13 )))[y6]} 
              {H8 :: (type_of (E9 ..) (T12 ..))[y6]} 
              {H7 :: (type_of (E10 ..) (arr (T12 ..) (T13 )))[y6]} 
              {H6 :: (type_of (E9 ..) (T12 ..))[y6]}
              [[. e_ref (arr T12 T13)]  : y6 . (H6 ..) = H2 , y6 . (H7 ..) = H3
              , y6 . (T12 ..) = T7 , y6 . (H8 ..) = H4 , y6 . (H9 ..) = H5 ,
              y6 . (T12 ..) = T8 ,  . T13 = T9 ,  . T13 = T10 ,
              y6 . (E9 ..) = E7 , y6 . (E10 ..) = E8 , y6 = z5 ]   => 
               [] e_ref T13 : (equal T13 T13)[]
            )
       )
  
  | {y7 :: (some [t : tp] block (x:exp. type_of x t))*}  {T18 :: tp[]} 
    {E12 :: exp[y7, E : exp]}  {T17 :: tp[]}  {T16 :: tp[]} 
    {H10 :: (type_of (E12 .. x) (T17 ))[y7, x : exp, x3 : type_of x (T18 )]}
    [[y7. t_lam (T18 ) (\E. E12 ..) (T17 ) (\x. \u. H10 .. x u)]  :  . T16 = T'
    ,  . arr T18 T17 = T , y7 . lam (T18 ) (\x1. E12 .. x1) = E , y7 = g ]   => 
      (case f of 
       | {z7 :: (some [t : tp] block (x:exp. type_of x t))*}  {T23 :: tp[]} 
         {E14 :: exp[z7, E : exp]}  {T22 :: tp[]}  {T21 :: tp[]} 
         {H12 :: (type_of (E14 .. x) (T22 ))[z7, x : exp, x3 : type_of x (T23 )]} 
         {H11 :: (type_of (E14 .. x) (T21 ))[z7, x : exp, x3 : type_of x (T23 )]}
         [[z7. t_lam (T23 ) (\E. E14 ..) (T21 ) (\x. \u. H11 .. x u)]  : z7, y9, z8 . (H12 ..) = H10
         ,  . arr T23 T21 = T16 ,  . T22 = T17 , z7, x9 . (E14 .. x9) = E12 ,
          . T23 = T18 , z7 = y7 ]   => 
           (case unique3 [z7, b : block (x:exp. type_of x (T23 ))]
                   <z7, z9. E14 .. z9.1> <. T22> <. T21>
                   ([z7, y10] H12 .. y10.1 y10.2)
                   ([z7, x10] H11 .. x10.1 x10.2) of 
            | {x8 :: (some [t : tp] block (x:exp. type_of x t))*} 
              {T26 :: tp[]}  {E15 :: exp[x8, E : exp]}  {T25 :: tp[]} 
              {H14 :: (type_of (E15 .. x) (T25 ))[x8, x : exp, x3 : type_of x (T26 )]} 
              {H13 :: (type_of (E15 .. x) (T25 ))[x8, x : exp, x3 : type_of x (T26 )]}
              [[. e_ref T25]  : x8, y11, z10 . (H13 ..) = H11 ,
              x8, z11, x11 . (H14 ..) = H12 ,  . T25 = T21 ,  . T25 = T22 ,
              x8, y12 . (E15 ..) = E14 ,  . T26 = T23 , x8 = z7 ]   => 
               [] e_ref (arr T26 T25) : (equal (arr T26 T25) (arr T26 T25))[]
            )
       )
  

## Type Reconstruction done: examples/unique/unique.bel  ##


\end{verbatim}

\end{document}

